\documentclass{beamer}
\usepackage{listings}
\usepackage{multicol}

\lstdefinestyle{all}{
  basicstyle=\ttfamily\small,
  showstringspaces=false,
  %% frame=tlrb                   
}

\lstdefinestyle{pesudocode}{
  style=all,
  basicstyle=\fontsize{9}{11}\ttfamily
}

\lstdefinestyle{erlang}{
  style=all,
  basicstyle=\fontsize{6.5}{7}\ttfamily
}

\setlength{\columnseprule}{1pt}
\def\columnseprulecolor{\color{black}}

\mode<presentation>{ \usetheme{boxes} }
\author{Anoop Aryal}

\AtBeginSection[] {
  \begin{frame}
    \frametitle{Outline}
    \tableofcontents[currentsection,currentsubsection]
  \end{frame}
}

\begin{document}

\begin{frame}
  \titlepage
  \frametitle{DHT in Erlang}
\end{frame}

\section{Background}
\begin{frame}
  \frametitle{Motivation}
  Recurring need for elastic sharding that can grow and shrink on-the-fly
  \begin{itemize}
  \item Maintainance
  \item Scale
  \item dynamic load balancing
  \item Growing/shrinking without pre allocating
  \end{itemize}
\end{frame}

\begin{frame}
  \frametitle{Language and Algorithm}
  Chord
  \begin{itemize}
  \item Consistent hashing - only K/n keys need to be remapped where K
    is the number of keys and n is the number of slots
  \item Ring of nodes that can come and go
  \end{itemize}

  Erlang
  \begin{itemize}
  \item Cheap threads
  \item OTP + Supervisors + multiple version of code in the VM: helps
    with high availability
  \item Message passing with location transparency
  \end{itemize}
\end{frame}


\section{Implementation}
\begin{frame}[fragile]
  \frametitle{The Chord Paper sample code}
  \begin{itemize}
  \item
    \begin{lstlisting}[mathescape, style=pesudocode]
$n$.join($n\prime$)
   predecessor = nil;
   successor = $n\prime$.find_successor($n$);
    \end{lstlisting}
    Object Oriented psuedocode. \emph{$n$} is the \emph{self}
    node. \emph{$n\prime$} is the other (remote) node we're
    joining. Note the location transparent call to
    \emph{find\_successor}.

  \item 
    \begin{lstlisting}[mathescape, style=pesudocode]
$n$.notify($n\prime$)
   if(predecessor is nil or $n\prime$ $\in$ (predecessor, n))
      predecessor = $n\prime$;
    \end{lstlisting}
    Use of mathematical notation $\in$ and (x,y), [x, y) for intervals
      etc.
  \end{itemize}
\end{frame}

\begin{frame}
  \frametitle{Goal} Keep the code as close to the paper as
  possible. That means solve for:
  \begin{itemize}
  \item Set operators and intervals
  \item Object Oriented convention
  \item Location transparency
  \end{itemize}
\end{frame}

\begin{frame}[fragile]
  \frametitle{Set Operators And Intervals}

  \begin{multicols}{2}
    \textbf{Paper}
   \begin{lstlisting}[mathescape, style=pesudocode]
$x \in (n, successor]$
    \end{lstlisting}

   \columnbreak

   
   \textbf{Erlang}
   
   Owing to the lack of infix operators, define a function like so:
   \begin{lstlisting}[mathescape, style=erlang]
between_oc(Num, {Low, High}) ->
   ...
   \end{lstlisting}
   Then, call it like so:
   \begin{lstlisting}[mathescape, style=erlang]
between_oc($X$, {N, Successor}).
   \end{lstlisting}
   
  \end{multicols}

  Would be nice if we could define infix operators and have mismatched
  parenthesis like $(x, y]$, $[x, y)$. But this is readable enough.

   
\end{frame}

\begin{frame}[fragile]
  \frametitle{OO be gone}

  Luckly, OO notation is only used for nodes and is used to primarily
  convey which node the function is running on. Essentially, taking
  \emph{Smalltalks} concept of sending a message to an object, and
  using that as sending a message to a node.

  \begin{multicols}{2}
    \textbf{Paper}
   \begin{lstlisting}[mathescape, style=pesudocode]
$n$.notify($n\prime$)
    \end{lstlisting}

   \columnbreak

   \textbf{Erlang}
    \begin{lstlisting}[mathescape, style=erlang]
notify($N$, $N\prime$) ->
   ...
    \end{lstlisting}
  \end{multicols}

  Effectively, we're borrowing \emph{self} from Python.
   
\end{frame}

\begin{frame}[fragile]
  \frametitle{Location Transparency}
  Building on \emph{self} and Erlangs template matching...

  \begin{multicols}{2}
    \textbf{Paper}
   \begin{lstlisting}[mathescape, style=pesudocode]
$n$.find_successor(Id)
    \end{lstlisting}
   This is sometimes used inside a ``for'' loop iterating over a
   ``finger table'' which, depending on how many nodes are present,
   could mean that $n$ is local or a remote node. It's hard to
   \emph{if/then/else} everywhere this occurs.

   \columnbreak

   \textbf{Erlang}
    \begin{lstlisting}[mathescape, style=erlang]
find_successor(SelfPid, SelfPid, Id) ->
   localOp();      
find_successor(_SelfPid, $N$Pid, Id) ->
   gen_server:call($N$Pid,
                   {find_successor, Id}).


handle_call({find_successor, Id}, _From, State) ->
   Reply = find_successor(self(), self(), Id),
   {reply, Reply, State};
...


%% location independent call
%% since we don't know if $N$ is local or remote
%% template matching will sort it out
find_successor(self(), $N$Pid, SomeId).  
    \end{lstlisting}
  \end{multicols}

  Here, we're adding another argument to functions after the
  \emph{self} argument to indicate where we want to actually run the
  function. We then use \emph{gen\_server:call/2} to remote it if the
  \emph{$N$Pid} doesn't match \emph{self}.
   
\end{frame}

\begin{frame}
  \frametitle{Further Adjustments}
  \begin{itemize}
  \item Because of in-order message handling of \emph{gen\_server},
    needed to use \emph{gen\_server:reply/2} from a spawned thread.
  \item Some gaps in the paper -- things that aren't defined. For
    example, key transfer is, more or less, assumed to be atomic in the
    paper. Doesn't specify what should happen if a \emph{get} is
    issued to a key in transit. It's left to the implementation.
  \end{itemize}
\end{frame}

\begin{frame}
  \frametitle{Memcached}

  To get a feel for the performance characterstics of the distributed
  hash table, I decided to implement a minimal \emph{Memcached} server
  and used the DHT as the data store. The DHT, in turn, used
  \emph{ETS} as the key value store. Implemented enough of the binary
  version of \emph{Memcached} protocol to make \emph{memslap} happy.

  \begin{itemize}
    \item Implemented: get, put, set
    \item Not Implemented: timed expiry. Rest is easy to implement,
      even the deferred/pipelined requests with \emph{flush}.
  \end{itemize}
\end{frame}


\section{Results}
\begin{frame}
  \frametitle{Slapped by \emph{memslap}}
  \begin{itemize}
  \item Erlang/DHT version is about 10x slower than
    \emph{Memcached}. (Not bad for about a month of commuter train
    hacking!)
  \item Erlang/DHT version can take far more connections and requests
    without erroring out than \emph{Memcached}!
  \end{itemize}
  (screencasts)
\end{frame}


\section{Future}
\begin{frame}
  \frametitle{Functional Programming + Math}

  After this experience with getting fairly close to being able to
  write distributed software by simply transcribing the paper,
  encouraged to look at other mathametically (predicate logic) proven
  algorithms and transcribing those. MDD - Mathametics Driven
  Development? It's not a new concept: ``Formal Methods'', Leslie
  Lamports TLA+.

\end{frame}
\begin{frame}
  \frametitle{Distributed Toolkit}

  Another direction is to start building distributed algorithm
  toolkit. Interested in the following algorithms so far:

  \begin{itemize}
  \item Paxos or Raft
  \item Amazon's Dynamo
  \item Distributed graph algorithms
  \item ...
  \end{itemize}
\end{frame}

\begin{frame}
  Interesting in collaborating with fellow journeymen on learning
  about the state of the art in distributed algorithms by implementing
  them. No particular end goal in mind.

  Github: https://github.com/aaryal/hive
  
  Thank you!
\end{frame}
\end{document}
