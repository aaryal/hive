\documentclass{beamer}
\usepackage{listings}
\usepackage{multicol}

\lstdefinestyle{all}{
  basicstyle=\ttfamily\small,
  showstringspaces=false,
  %% frame=tlrb                   
}

\lstdefinestyle{pesudocode}{
  style=all,
  basicstyle=\fontsize{9}{11}\ttfamily
}

\lstdefinestyle{erlang}{
  style=all,
  basicstyle=\fontsize{9}{11}\ttfamily
}

\setlength{\columnseprule}{1pt}
\def\columnseprulecolor{\color{black}}

\mode<presentation>{ \usetheme{boxes} }
\author{Anoop Aryal}

\AtBeginSection[] {
  \begin{frame}
    \frametitle{Outline}
    \tableofcontents[currentsection,currentsubsection]
  \end{frame}
}

\begin{document}

\begin{frame}
  \titlepage
  \frametitle{DHT in Erlang}
\end{frame}

\section{Background}
\begin{frame}
  \frametitle{Motivation}
  Recurring need for elastic sharding that can grow and shrink on-the-fly
  \begin{itemize}
  \item Maintainance
  \item Scale
  \item dynamic load balancing
  \item Growing/shrinking without pre allocating
  \end{itemize}
\end{frame}

\begin{frame}
  \frametitle{Language and Algorithm}
  Chord
  \begin{itemize}
  \item Consistent hashing - only K/n keys need to be remapped where K
    is the number of keys and n is the number of slots
  \item Ring of nodes that can come and go
  \end{itemize}

  Erlang
  \begin{itemize}
  \item Cheap threads
  \item OTP + Supervisors + multiple version of code in the VM: helps
    with high availability
  \item Message passing with location transparency
  \end{itemize}
\end{frame}


\section{Implementation}
\begin{frame}[fragile]
  \frametitle{The Chord Paper sample code}
  \begin{itemize}
  \item
    \begin{lstlisting}[mathescape, style=pesudocode]
$n$.join($n\prime$)
   predecessor = nil;
   successor = $n\prime$.find_successor($n$);
    \end{lstlisting}
    Object Oriented psuedocode. \emph{$n$} is the \emph{self}
    node. \emph{$n\prime$} is the other (remote) node we're
    joining. Note the location transparent call to
    \emph{find\_successor}.

  \item 
    \begin{lstlisting}[mathescape, style=pesudocode]
$n$.notify($n\prime$)
   if(predecessor is nil or $n\prime$ $\in$ (predecessor, n))
      predecessor = $n\prime$;
    \end{lstlisting}
    Use of mathematical notation $\in$ and (x,y), [x, y) for intervals
      etc.
  \end{itemize}
\end{frame}

\begin{frame}
  \frametitle{Goal} Keep the code as close to the paper as
  possible. That means solve for:
  \begin{itemize}
  \item Object Oriented convention.
  \item Location transparency
  \item Set operators and intervals
  \end{itemize}
\end{frame}

\begin{frame}[fragile]
  \frametitle{OO be gone}
  Luckly, this is only used for nodes.

  \begin{multicols}{2}
    \textbf{Paper}
   \begin{lstlisting}[mathescape, style=pesudocode]
$n$.notify($n\prime$)
    \end{lstlisting}

   \columnbreak

   \textbf{Erlang}
    \begin{lstlisting}[mathescape, style=erlang]
notify($n$, $n\prime$) ->
   ...
    \end{lstlisting}
  \end{multicols}

  Effectively, we're borrowing from Python with it's \emph{self}.
   
\end{frame}

\section{Future}

\end{document}
